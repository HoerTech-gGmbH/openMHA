%%% This file is part of the Open HörTech Master Hearing Aid (openMHA)
%%% Copyright © 2017 2018 HörTech gGmbH
%%%
%%% openMHA is free software: you can redistribute it and/or modify
%%% it under the terms of the GNU Affero General Public License as published by
%%% the Free Software Foundation, version 3 of the License.
%%%
%%% openMHA is distributed in the hope that it will be useful,
%%% but WITHOUT ANY WARRANTY; without even the implied warranty of
%%% MERCHANTABILITY or FITNESS FOR A PARTICULAR PURPOSE.  See the
%%% GNU Affero General Public License, version 3 for more details.
%%%
%%% You should have received a copy of the GNU Affero General Public License, 
%%% version 3 along with openMHA.  If not, see <http://www.gnu.org/licenses/>.

\section{A simple example configuration and how to start it}%
\label{sec:scenarios}%
\index{example configuration}%
\index{configuration!example}%
\index{configuration file}%

In this section, an example is shown of how to configure and start the
\mha{}. A simple algorithm is designed that applies a gain to the two-channel audio input.
Then examples are given on how to start this algorithm in different situations (\mhad{} in command line, \Matlab{} processing).

\subsection{Gain application}%
\label{sec:dyncmp}%

The example in this section describes how to apply a simple gain to the input signal.
It is meant to serve as an illustrative case.\footnote{
  The corresponding configuration file for this setup can be found at
  \emph{mha/examples/00-gain/gain.cfg}.
}
%
In this example the host plugin (the plugin that loads all other
plugins) will be \verb!mhachain!.
%
We first configure general parameters, such as the number of input 
channels, fragment size and the sampling rate:
\begin{verbatim}
nchannels_in = 2
fragsize = 64
srate = 44100
\end{verbatim}
%
We then load the host plugin and specify the audio input-output backend:
in this case we will be using audio files:
\begin{verbatim}
mhalib = mhachain
iolib = MHAIOFile
\end{verbatim}
%
Next, the host plugin \verb!mhalib! loads the \verb!gain! plugin
to perform the signal processing:
\begin{verbatim}
mha.algos=[gain]
\end{verbatim}
%
In the next few lines, we configure the parameters of the gain
plugin, setting the maximum allowed values and the gain values to be applied to the first and the second audio channel. 
%

\begin{verbatim}
mha.gain.min=-20
mha.gain.max=20
mha.gain.gains=[ -10 10 ]
\end{verbatim}

%
Finally, we specify the input and output audio file names to be processed as 
follows:
\begin{verbatim}
io.in = 1speaker_diffNoise_2ch.wav
io.out = 1speaker_diffNoise_2ch_OUT.wav
\end{verbatim}
The configuration we described so far is provided with this release 
in the file \emph{mha/examples/00-gain/gain.cfg}.
Now that we created a complete \mha{} algorithm configuration file,
in the next step, we want to start an \mha{} framework with this configuration.
%
In the terminal, change the working directory to the
\verb!mha/examples/00-gain!
directory of the \mha{}.
All binaries and libraries of the \verb!openMHA! should be in their respective
search paths, see \emph{README.md}).
Then \mha{} processing can be started with
\begin{verbatim}
mha ?read:gain.cfg cmd=start cmd=quit
\end{verbatim}
%
This will process the file \emph{1speaker\_diffNoise\_2ch.wav} and output 
\emph{1speaker\_diffNoise\_2ch\_OUT.wav}.
%\subsection{Using the \mhad{} and \Matlab{}}

%The \mhad{} configuration interface can be easily accessed from the
%\Matlab{} prompt through a generic graphical user interface,
%'mhagui\_generic' (see also \secpageref{sec:mhagui_generic}).
%This interface works on \Linux{} and \Windows{}. To start the \mhad{} on \Linux{}, type
%\begin{verbatim}
%MHA_LIBRARY_PATH=./
%LD_LIBRARY_PATH=./
%mha
%\end{verbatim}
%in the MHA 'bin' directory; on \Windows{} just double-click the file
%'mha.exe'. Then start \Matlab{}, and add the MHA matlab directory to
%your \Matlab{} search path using 'addpath'. Finally, before opening the user interface, add \verb!mhactl_java.jar! into your \Matlab{} classpath. To open the user interface, type
%\begin{verbatim}
%mhagui_generic
%\end{verbatim}
%at the \Matlab{} prompt.

\subsection{Starting the \mhad{} with JACK input/output}%
\label{sec:example_jack}%
\index{JACK}%
\index{framework configuration}%
\index{configuration!framework}%

In this section we will use the configuration settings described in
the previous section while using the JACK server as the audio
backend.\footnote{To run the JACK server reliably with small buffer
  sizes, you may need to optimize your operating system for low-delay
  audio. In the case of Ubuntu, you can install and boot a lowlatency
  linux kernel and answer "yes" to the question "Enable realtime process
  priority?" when installing jackd2, and add your user to the group
  "audio".
}
%
While in the previous section
we could  set the number of 
input channels, fragment size and sampling rate of the framework freely,
when using JACK the fragment size
and sampling rate have to match the values used by the JACK server.
%
Here we assume that the JACK server runs at sampling rate 44100 with a
fragment size of 64 samples.
%
Later we will show how to use a double buffer for those
situations, where it is not possible to start JACK with the desired
fragment size. 

The complete configuration for this example can be found in
\emph{mha/examples/00-gain/gain\_live.cfg}.
In the previous section we have specified which libraries to use
(plugins and the IO backend, which was \verb!MHAIOFile!). In this section 
we want to change IO backend to JACK. 
%
To do this, we modify the following line:
\begin{verbatim}
# IO plugin library name
iolib = MHAIOJack
\end{verbatim}
%
As in the previous section, it is assumed, that the environment
variables for finding executables and shared libraries are configured
properly (see \verb!README.md!).
The JACK server needs to be told which hardware input and output ports should be
connected to the mha:
\begin{verbatim}
io.con_in = [system:capture_1 system:capture_2]
io.con_out = [system:playback_1 system:playback_2]
\end{verbatim}
You may need to replace the ports with the port names used
by your system.
When the JACK server is runnng, execute \emph{jack\_lsp} in your command
line for a list of available clients and ports.
%
Then \mha{} live-processing can be started with
\begin{verbatim}
mha ?read:gain_live.cfg cmd=start
\end{verbatim}

\subsection{Adjusting the fragment size}%
\index{fragment size}%
\index{double buffering}%

If the required fragment size is not supported by the audio hardware,
double buffering can be used in the MHA frameworks. We assume now,
that the JACK server was started with a fragment size of 256 samples at
a sampling rate of 44.1 kHz. A fragment size of 64 samples in the algorithm 
can be reached by inserting a double buffer plugin between the framework and the algorithm. This is
done by replacing the MHA library \verb!mhachain! by the double
buffer plugin \verb!db!, which will load \verb!mhachain! as a
client. In the framework configuration the line \verb!mhalib = mhachain! needs
be replaced by
\begin{verbatim}
mhalib = db
mha.plugin_name = overlapadd
mha.fragsize = 64
\end{verbatim}
The fragment size of the framework will be set to that of the JACK
server, so \verb!fragsize = 64! on the top level needs to be replaced by
\verb!fragsize = 256!. The hierarchy of layers now changes (see 
file \emph{gain\_live\_double.cfg}, where between the framework
and \verb!mhachain! lies below the \verb!db! plugin.


\subsection{The MHA network interface}%
\label{sec:example_network}%
\index{framework configuration}%
\index{configuration!framework}%
Sometimes it is practical to change the MHA configuration during runtime. To
this end, \mha{} accepts configuration and control over a network connection.
%
When no command line parameters are given, the default port number
33337 and the loopback network interface 127.0.0.1 is used, i.e., only
connections from the local host are accepted (see section
\ref{sec:linuxmhaserver} on page \pageref{sec:linuxmhaserver} for
details).
%
To enter \mha{} commands, start a network client, e.g.\ Netcat, to
open a MHA console:
\begin{verbatim}nc localhost 33337\end{verbatim}

To read the framework configuration file, type
\begin{verbatim}
?read:mha/examples/00-gain/gain_live.cfg
\end{verbatim}
followed by the return key. If everything went well, the MHA will print
\verb!(MHA:success)!. In case of an error message \verb!(MHA:failure)!, MHA 
will also indicate the line containing the error. You need to correct this 
error using an editor you prefer and  reload it. If you receive an error 
message, \verb!(mha_parser) The variable is locked!, you need to close the 
\mhad{} and relaunch it. If the JACK server was not started yet, this is the
right moment to start your JACK server with the correct settings. One can use 
for example the \verb!QjackCtl! client to set the correct settings.
Make sure that the fragment size and sample rate of the JACK
sound server matches the MHA fragment size (see below if it
doesn't). After having successfully started the JACK server, the MHA
can be started by typing
\begin{verbatim}
cmd = start
\end{verbatim}
at the MHA console. The processing can be stopped at any time by
typing \verb!cmd = stop!.

Now, we want to access the variables of the algorithm. The easiest way
is to type \verb!?!, followed by the return key, in the console. This
will show the complete MHA configuration, including all framework
variables and plugin configuration. Usually, this
produces so much output, that the console has to be scrolled to see
the complete information. If only a subset or a single variable is of
interest, the prefix of that subset or variable can be put before the
\verb!?!, e.g.\ all variables of the processing chain can be reached
by typing \newline\verb!mha?!, the gain data in dB gains
by typing \newline\verb!mha.gain.gains?!. All monitor variable contents can
be stored into the file "example.mon" by typing
\verb!?savemons:example.mon!. The \mhad{} will be closed
by typing \verb!cmd = quit!.

\subsection{Start the MHA for real-time processing with \Octave{}/\Matlab{} from Linux}%
\index{Matlab}%
Alternatively the \mhad{} can be controlled from within Matlab or Octave.
%
The \mha{} \Matlab{} tools directory (\emph{./mha/tools/mfiles/}) has to
be in the \Matlab{} path and the mha java class must have been added by executing
\begin{verbatim}
> javaaddpath ./mha/tools/mfiles/mhactl_java.jar
\end{verbatim}
in your \Matlab{} command window.
%
We assume that your \Matlab{} process is running on the same host and
as the same user as the \mha{}, and that the \mhad{} runs with the
default port number 33337.
%
Now, communication with the configuration interface of the MHA is
possible through MHA \Matlab{} functions (see \secpageref{sec:matlab}
for a detailed documentation):
%
First create a MHA connection handle for the \Matlab{} tools by
typing
\begin{verbatim}
> h = struct( 'port', 33337, 'host', 'localhost' );
\end{verbatim}
at the \Matlab{} prompt. Then this handle can be used to connect to
the MHA:
%
\begin{verbatim}
> mha = mha_get( h, '' );
\end{verbatim}
%
The complete MHA configuration hierarchy is converted into a \Matlab{}
'struct' variable.
%
A configuration file\index{framework
configuration}\index{configuration!framework} of the \mhad{} can be
read by issuing an \mha{} query command:
%
\begin{verbatim}
> mha_query(mha,'','read:mha/examples/00-gain/gain_live.cfg');
\end{verbatim}
%
Note that the mha\_get command has to be repeated to read in the changed configuration.
%
Processing is then started by issuing the start command:
\begin{verbatim}
> mha_set( mha, 'cmd', 'start' );
\end{verbatim}
When the \mha{} processing is not needed any more, the MHA can be shut
down by calling
%
\begin{verbatim}
> mha_set( mha, 'cmd', 'quit' );
\end{verbatim}
%%% Local Variables: 
%%% mode: latex
%%% TeX-master: "openMHA_application_manual"
%%% indent-tabs-mode: nil
%%% coding: utf-8-unix
%%% End:
