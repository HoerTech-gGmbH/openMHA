\documentclass[11pt,a4paper,twoside]{article}
\usepackage{openMHAdoxygen}
\setlength{\headheight}{13.6pt}
\begin{document}
\MHAtitle{The control surface 'mhacontrol'}
\newpage
\MHAcopyright{}
\newpage
\tableofcontents
\newpage
\renewcommand{\leftmark}{\rightmark}
\pagenumbering{arabic}

\section{Introduction}

The Master Hearing Aid (MHA) is an algorithm development and evaluation
system.
%
The real-time system can be configured to behave as a hearing device.
%
The program 'mhacontrol' provides a control surface and fitting
interface.
%
This interface is intended to be usable without technical insight into
the underlying algorithms.

This manual will describe the dialogs of the control surface and
explains typical work flows.

\section{Starting 'mhacontrol'}

The control surface 'mhacontrol'\index{mhacontrol} is a Matlab
application.
%
To start the control surface, please start Matlab, change into the
directory with the application files, and type
%
\begin{verbatim}
mahcontrol hostname
\end{verbatim}
%
\verb!hostname! is the hostname\index{hostname} or IP address\index{IP
  address} of the MHA.
%
If the MHA runs on the same host as Matlab the hostname or IP address
can be omitted.
%
If the MHA and Matlab do not run on the same host, please make sure
that the firewall allows access to port\index{port}\index{firewall}
33337 (or the port used by the actual MHA configuration), and that the
MHA on the remote host is started listening on the appropriate network
devive (e.g., \verb!mha -i 0.0.0.0!).
%
A TCP network connection between MHA and mhacontrol is required.

Please note that the control tool will overwrite many values of the
MHA configuration files with values stored in the configuration of
mhacontrol (stored in Matlab files on the control PC).
%
This primarily affects the client data, i.e. configuration of dynamic
compression and fine-tuning, and the calibration parameters.
%
If the MHAIOJack audio backend is used, the Jack connections might be
changed by the control surface.

\section{Dialogs}

\subsection{The main screen 'mhacontrol'}\index{mhacontrol}\index{main screen}

The main screen (\figref{mhacontrol_main.png}) is divided into four regions:
%
In the top left area, a panel for algorithm selection\index{algorithm
  selection}\index{program selection} is provided, if the underlying
MHA contains multiple algorithms or programs.
%
An algorithm\index{algorithm selection}\index{program selection} can
be selected by clicking on an entry of the list.
%
The entry '(none)' mutes the output of the hearing aid.

In the upper right area is a panel with starter buttons for typically
used actions.
%
If a dynamic compression is available, a fitting interface (see
Section \ref{sec:fitting}) can be opened through the 'Fitting' button.
%
The 'In-situ signals' button opens a dialog for playback of in-situ
signals (see \ref{sec:insitu}).
%
A level meter can be opened by clicking 'Levels' (see Section
\ref{sec:levels}).

Below the 'Actions' panel is the 'Administration'
panel\index{administration} with start buttons for actions which are
needed only for the maintainance of the MHA setup.
%
These are 'Audio connections' (Section \ref{sec:audioconnections}) for
a selection of input and output sound card connections.
%
The calibration of the system can be accessed through 'Calibration'
(see Section \ref{sec:calibration}).
%
The generic MHA configuration browser can be reached through 'Advanced
control' (see Section \ref{sec:advanced}).

In the bottom left area some status informations\index{status
  information} are displayed, like selected
calibration\index{calibration ID}, current client ID\index{client ID},
gain prescritpion rule\index{gain rule} and current gain/fine-tuning
preset\index{preset}.

\MHAfigure[][\linewidth]{Main screen}{mhacontrol_main.png}

\subsection{Client selection and fitting}\label{sec:fitting}

\subsubsection*{Client database}

The 'mhacontrol' software contains a simple database for client management\index{client database}\index{database}.
%
Client entries can be created and modified in the client database
dialog, see \figref{mhacontrol_client_database.png}.
%
A client entry consists of first- and lastname, birthday, an unique
client ID\index{client ID}, and an arbitrary number of
audiograms\index{audiogram}.
%
In the top left panel a list of existing clients is shown.
%
A client can be selected for editing an audiogram entry by a single
mouse click on the list entry.
%
To select a client (with an audiogram selection) for fitting, please
click the 'select' button on the bottom of the screen.
%
With the buttons 'New client' and 'Edit client', the client data can
be entered or edited.

\MHAfigure{Client database dialog.}{mhacontrol_client_database.png}

On the right side of the dialog audiogram data can be managed.
%
Available audiograms are listed in the top right panel.
%
A new audiogram can be created by either manually entering the
audiogram data (see \figref{mhacontrol_audiogram_entry.png}, button
'Edit'), or by starting the in-situ audiometer
(\figref{mhacontrol_audiometer.png}, button 'Audiometer').
%
The selected audiogram can be printed using the 'Print' button.

\subsubsection*{In-situ Audiometer}%
\index{audiometry}\index{audiometer}\index{in-situ audiometry}\index{audiogram}\index{hearing threshold}\index{HTL}\index{UCL}

The MHA can provide in-situ audiometry through the audiometer dialog
(\figref{mhacontrol_audiometer.png}).
%
Using in-situ audiometry requires a properly calibrated MHA.
%
Even with a calibrated MHA, differences between in-situ- and
standard-audiometry are possible.
%
Currently, the MHA can measure the hearing threshold; UCL
measurement is not supported.

The audiometer can be controlled with the keyboard or with the
mouse/touch screen.
%
In the top left corner three buttons allow closing and saving of the
audiometer ('Quit'), cancelling of the current measurement ('Cancel')
and printing of the current audiogram ('Print').
%
Below, the audiogram of the right and left ear is shown.
%
Between the audiogram panels status information about the current test
signal is shown (ear, frequency, level and playback status).
%
In the bottom half of the dialog are buttons for selection of
right/left ear ('R', 'L'), for cursor navigation and data confirmation
($\leftarrow$, $\rightarrow$, $\uparrow$, $\downarrow$, 'Ok') and for
toggle of playback ('Sound on/off').
%
To find out the key short-cut please move the mouse pointer over the
buttons to see tool-tip information.

After closing and saving the audiogram ('Quit'), an audiogram ID can
be entered.
%
If you choose an existing audiogram ID, the audiogram with that ID will
be replaced.\index{audiogram ID}

\MHAfigure{In-situ audiometer dialog.}{mhacontrol_audiometer.png}

\subsubsection*{Audiogram editor}

An audiogram can be manually edited using the audiogram editor (see
\figref{mhacontrol_audiogram_entry.png}).\index{audiogram editor}\index{hearing threshold}\index{HTL}\index{UCL}
%
In the upper half of the dialog, the hearing threshold and
uncomfortable level can be entered for the right and left ear.
%
To skip a frequency, a value of 'inf' or 'nan' can be entered.
%
Below, on the left side, an audiogram ID can be entered.
%
The audiogram editor can be closed by hitting the 'Finish' button.
%
The current audiogram is displayed in the lower right side of the
dialog.

\MHAfigure{Audiogram editor.}{mhacontrol_audiogram_entry.png}

\subsubsection*{Print audiogram}

The current audiogram can be printed using a system printer.\index{print audiogram}

\subsubsection*{Fitting dialog}

The fitting dialog (see \figref{mhacontrol_fitting_tool.png}) is
divided into three columns. The middle column contains control
elements to select a gain prescription rule and create a first
fit. Below is a box for preset management. A preset contains all
settings which can be made in this dialog, including gain prescription
rule and fine-tuning, for a specific client and algorithm.  The
columns on the left and right side of the screen contain the
ear-specific settings for the right and left ear, respectively. In the
graph the nominal target level of a LTASS at three different levels is
shown with thick solid lines, calculated in third-octave bands. The
thin dashed lines mark the corresponding input level. The shaded areas
mark the ISO-226 normal hearing threshold (gray) and the threshold of
the selected subject (green). Below the graph controls for fine-tuning
can be found: An additional gain and a gain limitation can be
configured in frequency bands, by selecting \verb!gain! or
\verb!maxgain! in the drop-down menu.

To create a first fit, first select the appropriate gain prescription
rule from the list. In MHA configurations with a unilateral
compressor, also the ear to be fitted must be selected\footnote{The
  selection of an ear in the fitting dialog does not affect any signal
  routing. A correct mapping of MHA input and output channels to the
  correct ear must be assured.}. Pressing the `Create First Fit'
button will create a first fit, upload it to the MHA and store it as a
preset in the preset list. The fine-tuning parameters are set to
default values.  A first fit can not be overwritten by a fine-tuned
preset.  Creating a new first fit (e.g. after selecting another
audiogram) will overwrite any previous first fit for the selected gain
rule.

\MHAfigure[][\linewidth]{Fitting dialog.}{mhacontrol_fitting_tool.png}

\subsection{In-situ signals}\label{sec:insitu}\index{in-situ signal}\index{audio file}\index{wave file}

Audio signals can be played back through the PHS, replacing the
microphone inputs.
%
The in-situ signal dialog (\figref{mhacontrol_in-situ_signals.png})
shows a list of available sound files, a list of check boxes for
channel selection, and control elements for the other configurable
parameters of the underlying \verb!addsndfile! plugin.
%
The number of microphone channels and the channel order depends on the MHA
configuration.
%
Input channels can be selected for playback by activating their check
box.
%
The playback RMS level and the meaning of the \verb!level! variable
depends on the settings in the \verb!levelmode!  field:
\verb!relative! means that the signal is scaled by the
\verb!level!-value relative to its original level, with 0~dB FS
corresponding to 94~dB SPL. If \verb!peak! is selected, the level
denotes the peak level of the input file, and with \verb!rms! the RMS
level is denoted. \verb!rms_limit40! denotes the RMS level, except if
the peak level exceeds RMS level plus 40~dB, then the peak level is
set to level+40~dB.

Please make sure that the sound files have the same sampling rate
which is used at the position of the 'addsndfile' plugin, i.e. take
account for internal re-sampling.

\MHAfigure[][0.8\linewidth]{Playback of in-situ sound files.}{mhacontrol_in-situ_signals.png}

\subsection{Level meter}\label{sec:levels}\index{level meter}\index{broadband level}\index{gain}\index{input level}\index{output level}

The level meter (\figref{mhacontrol_levels.png}) contains two areas.
%
In the upper panel, the broad band level at input and output is shown.
%
Below that panel, the time constant for RMS level averaging can be
adjusted (broad band level meter only).

In the bottom half of the dialog, the input level and gain of the
dynamic compression is shown as a function of center frequency.

\MHAfigure{Level and gain meter.}{mhacontrol_levels.png}

\subsection{Audio connections}\label{sec:audioconnections}\index{connections}\index{JACK}\index{audio backend}

The 'mhacontrol' application can provide a dialog for connecting input
and output channels to specific hardware audio channels, see
\figref{mhacontrol_audio_connections.png} for a screenshot (this
requires the JACK audio backend).
%
Each MHA input channel can be connected to a hardware input, and
each output channel can be connected to a hardware output.
%
Connections can be saved in presets that are listed in the left panel of the connection dialog.
%
A preset can be selected by clicking on its entry in the list.
%
Please note that in the connection dialog only connections to hardware
ports are supported.
%
If you need connections to software ports, please use JACK connection
tools on the MHA computer.
%

\MHAfigure[][0.8\linewidth]{Audio connection manager.}{mhacontrol_audio_connections.png}

\subsection{Calibration}\label{sec:calibration}\index{calibration}

The calibration dialog (\figref{mhacontrol_calibration.png}) gives access
to the calibration of the PHS.
%
Calibrations can be recalled by selecting an entry in the calibration
list.
%
A calibration consists of a broadband calibration value for each
microphone, a broadband calibration value for the receiver and a
frequency equalization of the receiver.
%
In the bottom half of the dialog, the broadband calibration values for
inputs and outputs are shown, and the frequency response of the
receiver equalization is plotted on the right part of the dialog.

The calibration dialog can be closed using the 'Close' button.
%
An unused calibration can be removed via the 'Remove' button.
%
To create a new calibration, e.g. for re-calibration of the system,
first select the base calibration from which the new calibration will
inherit its parameters.
%
Then press the 'New' button.
%
A calibration wizard will be started.

\MHAfigure{}{mhacontrol_calibration.png}

\subsubsection*{Calibration wizard}

The calibration wizard guides through the calibration process.
%
On the first page (\figref{mhacontrol_calibration_wiz1.png}), the
calibration ID of the new calibration can be entered.\index{calibration ID}
%
Below, you can decide whether you would like to calibrate all input
channels in parallel, or one after each other.
%
It is also possible to skip input or output calibration.
%
To start with calibration of the output channels (or calibration of
input channels, if output channels are skipped), press the 'Start'
button.
%
It is possible to exit the calibration process at any time by clicking
'Cancel'.

\MHAfigure{Calibration wizard, start page.}{mhacontrol_calibration_wiz1.png}

For calibration of an output channel a speech-shaped noise signal is
played back.\index{output calibration}
%
First, select the appropriate receiver response.
%
It is assumed that the output level is 80 dB SPL; if the measured
level differs from 80 dB, please use the green or red buttons to
decrease or increase the level.
%
If you use a coupler for measuring the level, please take the coupler
correction for the speech shaped noise test signal into account.
%
If the output level of 80 dB SPL is reached, press 'Next' to proceed
with the next output channel, or with the input channels if all output
channels are calibrated.

\paragraph{Important note}
%
During the calibration procedure, a speech shaped noise with an
internal free field level of 80~dB SPL is played.
%
If the combination of selected receiver correction and transducer type has a flat frequency response (e.g., free field loud speaker), then the external level meter should display 80~dB SPL.
%
In a typical approach, the receiver correction does not aim to produce
a flat coupler reponse, but also includes the
real-ear-to-coupler-difference (RECD) and the real-ear-unaided-gain
(REUG).
%
For internally generated speech-shaped noise signal, typical level
corrections are in the range of 0 to 10 dB.
%
With a 2~cm\textsuperscript{3} coupler (IEC 60318-5:2006), the expected
coupler level is 78.5~dB SPL.
%
%For the Br\"uel \& Kj\ae{}r ear simulator type 4157 and ear mould
%simulator type DP 0370, the expected coupler level is 87.9~dB SPL.
%

\MHAfigure{Calibration wizard, calibration of output channel.}{mhacontrol_calibration_wiz2.png}

To calibrate an input channel, or all input channels simultaneously,
the headsets are placed onto an artificial head.
A speech shaped noise is presented with a level of 80
dB SPL at the prospective position of the artificial head's center.
The artificial head is then put into this position and adjusted so that it 
faces the the loudspeaker exactly. \index{input calibration}\index{microphone calibration}
%
See \figref{mhacontrol_calibration_wiz3.png} for a screen shot of the
calibration wizard page.
%
If the MHA provides a loud speaker output which has been
calibrated properly in the previous step, it can be used for playback
of the speech shaped noise.
%
When you press 'Next', the level is measured during 2 seconds, then
the next screen is displayed.

\MHAfigure{Calibration of input channels.}{mhacontrol_calibration_wiz3.png}

In the last stage (see \figref{mhacontrol_calibration_wiz4.png}), the
calibration values and receiver response can be verified.
%
If you press 'Finish', the calibration based on the previous
calibration steps will be uploaded to the MHA and stored in the
calibration database.

\MHAfigure{Calibration wizard, final page.}{mhacontrol_calibration_wiz4.png}

\subsection{Advanced control}\label{sec:advanced}\index{advanced control}\index{mhagui\_generic}

A generic parameter control interface to the MHA is available. See
\figref{mhacontrol_advanced.png} for a typical root level dialog.
%
A description of this generic interface can be found in the MHA
manual.

\MHAfigure{Generic MHA parameter interface.}{mhacontrol_advanced.png}

\section{Appendix}

\subsection{MHA backend plugins}

These plugins are automatically located by the mhacontrol interface:
%
\begin{itemize}
\item altplugs
\item audiometerbackend
\item addsndfile
\item transducers
\item dc\_simple
\item finetuning
\item fftfilterbank
\end{itemize}

\bibliography{MHA}

\printindex

\end{document}

% LocalWords:  audiogram MHA LTASS
