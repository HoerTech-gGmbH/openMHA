\documentclass[11pt,a4paper,twoside]{article}
\usepackage{mha}
\begin{document}
\MHAtitle{HearCom WP9: Correction of the Siemens Acuris P receiver frequency response in the MHA}
\newpage
\MHAcopyright{}
\newpage
\tableofcontents
\newpage
\renewcommand{\leftmark}{\rightmark}
\pagenumbering{arabic}

\section{Introduction}

Calibration of a hearing aid is a non-trivial task.
%
The concept of the MHA calibration is as follows:
%
Within the MHA the signal is available as free field sound pressure
level in Pascal.
%
To obtain these levels, the frequency response of the microphones has
to be corrected, and a broadband scaling is applied to the microphone
signal.
%
Since internally all levels are assumed to be free field level, the
receiver response has to be equalized at the output of the MHA, and a
broad band gain is applied to the signal to convert back from sound
pressure level to DA-converter levels.
%
In this documentation the procedure to receive a receiver correction
is described.
%
The correction of the microphone response is less complex, since it
does not depend on the individual ear plug and ear canal.
%
Additionally, most microphones can be assumed to have a flat frequency
response in a large frequency range.
%
This often allows to ommit the correction of the microphone response
in the MHA calibration.
%
The broadband calibration of the MHA is described in the 'mhacontrol'
control surface manual.

\section{Determining the receiver response}

The average normal hearing threshold for free field representation is
defined in ISO 226.
%
An ideal system (and any audiometer) must be able to reproduce this
threshold.
%
Thus a system which can reproduce the normal hearing threshold can be
assumed to have a flat receiver response.
%
This defines a way to receive a receiver correction:
%
The hearing threshold of a large number of normal hearing subjects is
recorded with the hearing aid receiver without equalization.
%
The difference between the recorded average normal hearing threshold
and the ISO normal hearing threshold defines the frequency dependent
gain to be applied to the receiver.

\paragraph{Note:} 
%
The resulting frequency response of the MHA system is not flat when
measured with a receiver coupler (e.g., B\&K ear simulator, see
below).
%
All data are valid for the system (amplifier, gain settings,
transducer type) used for the measurement of the thresholds.
%
A correction for the binaural advantage of 2~dB has been taken into
account in the broadband calibration value.


\section{In-situ audiometry without receiver correction}

For measurement of the average hearing threshold on the system without
receiver correction, the pure tone audiogram eight normal hearing
subjects was recorded using the MHA in-situ audiometer.
%
All data are gathered with the Siemens Acuris P headset attached to
the Behringer ADA8000 DA-converter, without receiver correction, and
at a broadband calibration value of 137.2~dB (sine wave with -10~dB
RMS relative full scale correspondes to an internal level of 127.2~dB
SPL).
%% new peaklevel: 137.2-13.5 = 1
%
The hearing threshold was measured at the standard audiometric
frequencies (125, 250, 500, 750, 1000, 1500, 2000, 3000, 4000, 6000
and 8000 Hz).
%
The step size was set to 2~dB, to allow a high level resolution.
%
At each level step, a hanning ramp of 125~ms was applied to the
signal.
%
In the audiometer display, the level relative to the ISO normal
hearing threshold was displayed (hearing level, dB HL).

Before averaging of the hearing thresholds, it was individually
corrected by the standard audiometry pure tone audiogram, to reduce
interindividual differences.
%
The individual results are shown in \figref{acuris_calib_aud_indiv}.
%
The median of these hearing thresholds as well as the 25\% and 75\%
quantiles are displayed in \figref{acuris_calib_aud_median}.
%
The median value was directly used as receiver correction.

\MHAfigure[][\linewidth]{Individual in-situ audiograms of 8 normal hearing test
  subjects (16 ears), individually corrected by their standard
  audiogram. The receiver correction was set to a flat response.
}{acuris_calib_aud_indiv}

\MHAfigure[][\linewidth]{Median of in-situ audiograms (thick solid
  line), with 25\% and 75\% quantile (dashed lines).
}{acuris_calib_aud_median}

\section{FIR coefficients}

The calibration wizard which is part of 'mhacontrol' creates an FIR
filter for receiver equalization, based on a coarsely sampled receiver
correction (here the median audiogram was taken as sampled receiver
correction).
%
For a minimal delay and better match of the absolute frequency
response, a minimal phase filter was designed:
%
First, a high resolution frequency response $X(f)$ (resolution 1~Hz)
was interpolated, using cubic spline interpolation.
%
Above 100~Hz the interpolation was performed on a logarithmic
frequency scale.
%
Below 100~Hz, the frequency response was interpolated on a linear
frequency scale, to avoid extreme numerical values near zero.
%
The length of the filter was chosen to match the block size $P$ of the
MHA setup, to allow an effective FFT-based convolution.
%
The minimal phase filter spectrum $H(f)$ is
%
\begin{equation}
  H(f) = |X(f)| e^{i \mathcal{H}\left\{\log|X(f)|\right\}},
\end{equation}
%
with the interpolated input response $X(f)$ and the Hilbert
transformation $\mathcal{H}\{\cdots\}$.
%
The impulse response was calculated by applying an inverse FFT to the
minimal phase filter spectrum.
%
The first $P+1$ taps of the impulse response where chosen for receiver
equalization, and windowed using a half hanning window, to avoid a
signal step at the end of the filter.

The resulting FIR filter coefficients with the corresponding frequency
response is shown in \figref{acuris_calib_behringer_irs}.
%
Note that the correction at 1~kHz was set to zero by shifting the
frequency response by 13.5~dB.
%
The broadband calibration factor was changed by the same amount from
137.2~dB to 123.7~dB in order to fully implement the desired
correction.
%
Additionally, a 2~dB correction for the binaural advantage in the
measurement of the ISO 226 thresholds has been applied to the
broadband calibration factor, resulting in a broadband calibration
factor 125.7~dB.

\MHAfigure{FIR filter coefficients (upper panel) and
  corresponding frequency response (lower panel, solid line) of the
  Acuris P receiver equalization connected to a Behringer
  ADA8000. With a dashed line and circles the median of the normal
  hearing threshold is plotted. The offset between the frequency
  response and the hearing threshold is corrected by the broadband
  calibration value.}{acuris_calib_behringer_irs}


\section{Coupler level}

To allow a verification of the calibration, the sound pressure level
in a coupler has been measured as a function of frequency.
%
A Br\"ul \& Kj\ae{}r Ear Simulator Type 4157 with the ear mould
simulator DP 0370 and a tube length of 25~mm has been used.
%
The hearing aid has been mounted on the ear simulator according to the
B\&K technical documentation, see
\figref{b_k_ear_simulator_ear_mould_simulator} for details.
%
Additionally, the 2~cm\textsuperscript{3} coupler level according to
IEC 60318-5:2006 has been measured.

\MHAfigure{B\&K ear and ear mould simulator (plot taken from technical
  information of the B\&K ear
  simulator).}{b_k_ear_simulator_ear_mould_simulator}

The level has been measured using a B\&K 2610 level meter.
%
In the calibrated MHA running the receiver correction for the Acuris P
and a Multiface II DA converter, a sine wave with an RMS level of
85~dB SPL free field has been created.
%
The coupler level was measured at three test frequencies per octave,
between 50 and 6400~Hz.

\figref{coupler_level_multiface} shows the measured coupler levels for
the ear and ear-mould simulator (dashed blue line with open circles),
and for the 2~cm\textsuperscript{3} coupler (solid black line with
dots). The data was measured using the Multiface II DA converter.
%

\MHAfigure[][\linewidth]{Coupler level for a calibrated MHA, displayed
  using the B\&K ear simulator 4157 and the B\&K ear mould simulator
  DP 0370 (dashed blue line with open circles). The level measured
  using the 2~cm\textsuperscript{3} coupler (IEC 60318-5:2006) is
  plotted with a solid black line and filled
  dots.}{coupler_level_multiface}

\section{Voltage at receiver input}

If a B\&K ear simulator is not available, the receiver correction can
be verified by measuring the voltage at the receiver, for a given
internal signal level.
%
In \figref{acuris_calib_receiver_voltage} the RMS voltage at the
receiver for a sine wave with an internal free field level of 85~dB
SPL is given.
%
During the measurement, the receiver correction was active.

\MHAfigure[][\linewidth]{Voltage in mV RMS at the receiver, at an
  internal sound pressure level of 85~dB free field. The voltage was
  measured while the receiver was mounted on the B\&K ear and ear plug
  simulator.}{acuris_calib_receiver_voltage}

\section{Frequency response of the PHS}

The frequency response from a free field loudspeaker to the head set
microphones mounted in a free field position (i.e, no HRTF) has been
measured.
%
The response was measured in a sound booth for free field testing.
%
The frequency responses are shown in \figref{response_ls_mics.png}.
%
The ripple is probably caused by the room acoustics of the sound booth.
%
A resonance at 5~kHz is visible for all microphones.

In \figref{response_phs.png} the frequency response from
the loudspeaker to the 2~cm\textsuperscript{3} coupler via the
calibrated PHS (red line) and the digital gain of the PHS (blue line)
is shown.
%
The high gain between 2.5 and 5~kHz reflects the REUG gain derived
from the normal hearing threshold measurement.

The frequency response measurements were performed with the 'Jack/Alsa
perceptual analyser' (japa) application developed by Fons Adriaensen.

\MHAfigure[][\linewidth]{Frequency response from loudspeaker to the
  front microphone (red line), middle microphone (blue line) and rear
  microphone (green line). On the $x$-axis the frequency in Hz is
  plotted, the $y$-axis shows relative gain in
  dB.}{response_ls_mics.png}

\MHAfigure[][\linewidth]{Frequency response from loudspeaker to the
  2~cm\textsuperscript{3} coupler via the calibrated PHS (red line),
  and digital gain of the PHS (blue line). On the $x$-axis the
  frequency in Hz is plotted, the $y$-axis shows relative gain in
  dB.}{response_phs.png}

\end{document}
